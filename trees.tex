
 
 Another specific aspect of our design is how we measure model {\em comprehensible}. 
We assume that software engineers will want to browse and understand and audit the results of any
 algorithm that has the presumption to tell them to change this our that.
 One way to present the results of an optimizer is to generate a {\em domination tree}, as fp;;pws"
 \bi
 \item
 Take all the examples
 ever evaluated by the optimizer; 
 \item
 Score each individual  by $|\Omega_I(x)|$, which is the number
 of other individuals dominated by $x$; 
 \item
 Use the CART decision tree learner~\cite{breiman1984classification} to summarize the
 decisions that lead to difference domiantion scores.
 \ei
  In
 this paper, we will say domination trees with {\em fewest} nodes and leaves
 are {\em more} comprehensible. 
 For example, Figures~\ref{fig:dom1} and ~\ref{fig:dom2} show two trees generated from
 the examples evaluated by FLASH and ePAL while optimizing the SS1 model of 
 Figure~\ref{fig:unconstrained_case_studies}.
 Note that one of these trees is far smaller  and hence easier
 to browse, understand, and audit.
 
  
 
 \begin{figure}[!b]
 \renewcommand{\baselinestretch{0.65}}
\hspace{0.2cm}\begin{lstlisting}[
escapechar=@,
basicstyle=\scriptsize\ttfamily,
keepspaces=true,
linewidth=3.25in,
frame=none,
numbers=right]
@\bh@sccp=0@\eh@
|    print_used_types=0
|    |    ipsccp=0 (6.5)
|    |    ipsccp=1
|    |    |    x[10]=0 (12) 
|    |    |    x[10]=1 (19) 
|    @\bh@print_used_types=1@\eh@
|    |    ipsccp=0 (14)
|    |    @\bh@ipsccp=1@\eh@
|    |    |    time_passes=0  (24)
|    |    |    @\bh@time_passes=1@\eh@
|    |    |    |    jump_threading=0 (28)
|    |    |    |    @\bh@jump_threading=1@\eh@ (31)
sccp=1
|    ipsccp=0 (1.5) 
|    ipsccp=1 (7.5) 
\end{lstlisting}
\caption{\small{A domination 
tree. Optimizer = FLASH; model= SS1, from Figure~\ref{fig:unconstrained_case_studies}.
Numbers in brackets (e.g. ``(31)'') show how many other individuals are dominated by individuals in a particular leaf.
For example, on the last line, there is a leaf whose individuals dominate (on average)
31 others.
The branch marked in gray leads to the branch with greatest score.
}}\label{fig:dom1}  
\end{figure}
 
 \begin{figure}[!b]
\renewcommand{\baselinestretch{0.65}}
\hspace{0.2cm}\begin{lstlisting}[
escapechar=@,
basicstyle=\scriptsize\ttfamily,
keepspaces=true,
linewidth=3.25in,
frame=none,
numbers=right]
ipsccp=0
|    iv_users=0
|    |    sccp=0
|    |    |    print_used_types=0
|    |    |    |    jump_threading=0
|    |    |    |    |    time_passes=0
|    |    |    |    |    |    instcombine=0 (16)
|    |    |    |    |    |    instcombine=1 (12)
|    |    |    |    |    time_passes=1
|    |    |    |    |    |    instcombine=0 (10)
|    |    |    |    |    |    instcombine=1 (11)
|    |    |    |    jump_threading=1
|    |    |    |    |    gvn=0
|    |    |    |    |    |    inline=0
|    |    |    |    |    |    |    simplifycfg=0 (22)
|    |    |    |    |    |    |    simplifycfg=1 (20)
|    |    |    |    |    |    inline=1 (26)
|    |    |    |    |    simplifycfg=0 (32)
|    |    |    |    |    simplifycfg=1 (33)
|    |    |    print_used_types=1
|    |    |    |    |inline=0 (47)
|    |    |    |    |inline=1 (43)
|    |    sccp=1
|    |    |    print_used_types=0 (1)
|    |    |    |print_used_types=1 (7)
|    iv_users=1
|    |    sccp=0
|    |    |    print_used_types=0
|    |    |    |    jump_threading=0 (30)
|    |    |    |    jump_threading=1 (42)
|    |    |    print_used_types=1
|    |    |    |    inline=0 (56)
|    |    |    |    inline=1 (62)
|    |    sccp=1
|    |    |    instcombine=0 (26)
|    |    |    instcombine=1 (33)
@\bh@ipsccp=1@\eh@
|    @\bh@sccp=0@\eh@
|    |    print_used_types=0
|    |    |    iv_users=0
|    |    |    |    jump_threading=0 (50)
|    |    |    |    jump_threading=1 
|    |    |    |    |    instcombine=0 (53)
|    |    |    |    |    instcombine=1 (54)
|    |    |    iv_users=1
|    |    |    |    simplifycfg=0 (59.5)
|    |    |    |    simplifycfg=1 
|    |    |    |    |    time_passes=0 (63)
|    |    |    |    |    time_passes=1 (66)
|    |    @\bh@print_used_types=1@\eh@
|    |    |    iv_users=0
|    |    |    |    time_passes=0 (69)
|    |    |    |    time_passes=1
|    |    |    |    |    instcombine=0 (73)
|    |    |    |    |    instcombine=1 (71)
|    |    |    @\bh@iv_users=1@\eh@
|    |    |    |    gvn=0 (76)
|    |    |    |     @\bh@gvn=1 (79)@\eh@
|    sccp=1
|    |    print_used_types=0
|    |    |    jump_threading=0 (3)
|    |    |    jump_threading=1 (16)
|    |    print_used_types=1
|    |    |    iv_users=0 (35)
|    |    |    iv_users=1 (50)
\end{lstlisting}
\caption{\small{Another domination tree. Same
format as Figure~\ref{fig:dom1} but the optimizer is ePAL.
This tree is larger than Figure~\ref{fig:dom1}
(and hence harder to understand) since ePAL evaluated more examples than FLASH.
}}\label{fig:dome2}  
\end{figure}
 
 
 